\chapter{Mezzi porosi e legge di Darcy}

I materiali porosi sono dei mezzi in cui si possono distinguere due parti: una matrice solida, la parte solida che costituisce la struttura del corpo, e lo spazio vuoto restante, che può essere riempito con uno o più fluidi. \\
Risulta spesso interessante studiare il moto di un fluido in un mezzo poroso in cui sia possibile distinguere delle fratture o dei canali al suo interno. Nell'analisi del moto di un fluido in una frattura tra diversi strati geologici, ad esempio,  si pu`o pensare che il fluido, oltre a insinuarsi e scorrere nella frattura, si propaghi mediante filtrazione negli strati adiacenti la frattura stessa.
Un'altra possibile applicazione in ambito idrogeologico riguarda lo studio di come i fiumi irrighino il terreno ad essi circostante. 
\par Il modo del fluido all'interno del mezzo poroso viene descritto tramite la legge di Darcy.  Generalmente le dimensioni delle fratture sono molto inferiori di quelle del dominio occupato dal mezzo poroso. Questo ha portato allo sviluppo di tecniche per lo studio del flusso all'interno delle fratture basate su modelli ridotti. Questa categoria di modelli si inserisce nell'ambito dei modelli multiscala, ed hanno il vantaggio di evitare la risoluzione del campo di moto su scale spaziali molto piccole nella frattura.

\section{Legge di Darcy}
La legge di Darcy descrive la filtrazione di un fluido incomprimibile all'interno di un mezzo poroso. In particolare costituisce un legame tra il campo di velocità del fluido e il gradiente di pressione nel mezzo poroso. \\
la legge ha la forma:
\begin{equation}
\vec{u} =- \frac{\vec{K}}{\mu}
\end{equation}



