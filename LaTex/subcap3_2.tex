\lstnewenvironment{Code}[1][]{\lstset{basicstyle=\small\ttfamily, columns=fullflexible, keywordstyle=\color{red}\bfseries, commentstyle=\color{blue},
language=C++, basicstyle=\small, numbers=left, numberstyle=\tiny, stepnumber=1, numbersep=5pt, frame=shadowbox, #1}}{}

\chapter{Implementazione delle fratture}

Il nostro progetto parte da un codice già esistente per la soluzione del problema di Darcy in un network di fratture con intersezione di tipo \textit{Cross }, ossia due fratture che si intersecano formando una \textit{X}. Noi abbiamo esteso il codice anche al caso di fratture con intersezione di tipo \textit{Bifurcation}, cioè tre fratture che si intersecano nello stesso punto formando una \textit{Y}.

Passiamo ora a descrivere le principali classi del codice che definiscono la struttura e le proprietà delle fratture, porgendo particolare attenzione alle classi modificate e a quelle da noi implementate.

\section{La Classe \texttt{FracturesSet}}
La class \texttt{FracturesSet} è la classe che contiente tutte le fratture e le rispettive intersezioni. La funzione che costruisce l'insieme delle fratture e le eventuali intersezioni è la funzione  \texttt{init}, che prende in input tutti i parametri necessari quali il numero di fratture, le mesh per l'integrazione e una variabile di tipo GetPot per la lettura dal file data. 

\begin{Code}[caption={Classe \texttt{FracturesSet}}]
class FracturesSet
{
public:

    enum
    {
        FRACTURE_UNCUT = 10000,
        FRACTURE_INTERSECT = 10000
    };

    FractureHandler ( const GetPot& dataFile,
                      const size_type& ID,
                      const std::string& section = "fractureData/" );

    void init ( );

    ...

private:
	FracturePtrContainer_Type M_fractures;

	FractureIntersectPtr_Type M_intersections;
	....
};
\end{Code}
La classe contiente l'insieme di tutte le fratture \texttt{M\_fractures}, una variabile di tipo vettore di puntatori al classe \texttt{FractureHandler} che verrà descritta dopo, e l'insieme di tutte le intersezioni \texttt{M\_intersections}, un puntatore alla classe \texttt{FractureIntersect}. 


\section{La Classe \texttt{FractureHandler}}

La classe  \texttt{FractureHandler} è la classe che inizializza e gestisce ogni frattura. I principali campi di questa classe sono:
\begin{itemize}
\item \texttt{M\_data}, classe \texttt{FractureData} che contiene tutte le informazioni circa la natura geometrica e fisica della singola frattura
\item \texttt{M\_levelSet}, puntatore alla classe \texttt{LevelSetHandler} di cui parleremo più avanti
\item i metodi di integrazione e le mesh per pressione e velocità
\item il vettore dei gradi di libertà estesi per velocità e pressione.
\end{itemize}

La particolarità di questa classe è che possiede due mesh: una mesh 2d \texttt{M\_meshMapped} e una mesh 1d \texttt{M\_meshFlat}. Questo deriva dal fatto che le fratture sono su un piano e hanno una rappresentazione del tipo $y=f(x)$, cioè i loro punti hanno coordinate $(x,y)$. La libreria che noi usiamo, \textit{GetFEM++}, non è in grado di integrare su una mesh i cui punti hanno coordinate $(x,y)$ un'equazione 1d come quella del nostro modello ridotto. La tecnica è quindi quella di risolvere il problema integrando sulle mesh " piatte " 1d ottenute proiettando le mesh reali, e successivamente interpolare i risultati ottenuti per ritornare sulle mesh 2d. La creazione delle mesh è nella funzione \texttt{ init}.
\par La funzione principale di questa classe è  \texttt{setMeshLevelSetFracture }.
Tale funzione crea un legame tra la frattura corrente e le fratture con cui ha un'intersezione, tenendo traccia della mesh e dei valori del levelset della frattura corrente nei punti della frattura intersecata. Nel caso di intersezione di tipo \textit{Cross}, dove l'intersezione tra due levelset non è detto che avvenga su due punti delle rispettive mesh, vengono aggiunti i gradi di libertà estesi, due per la velocità e uno per la pressione. Nel caso dell'intersezione di tipo \textit{Bifurcation} l'introduzione di tali elementi non si rende necessaria.


\begin{Code}[caption={Funzione \texttt{setMeshLevelSetFracture}}]
size_type FractureHandler::setMeshLevelSetFracture (
FractureHandler& otherFracture, size_type& globalIndex, const std::string& type )
{	
    const size_type otherFractureId = otherFracture.getId();
    size_type numIntersect = 0;
    
    if ( !M_meshLevelSetIntersect[ otherFractureId ].get() )
    {
        M_meshLevelSetIntersect[ otherFractureId ].reset ( new GFMeshLevelSet_Type ( M_meshFlat ) );
        LevelSetHandlerPtr_Type otherLevelSet = otherFracture.getLevelSet();
        M_levelSetIntersect [ otherFractureId ].reset ( new GFLevelSet_Type ( M_meshFlat, 1, false )  );
        M_levelSetIntersect [ otherFractureId ]->reinit();

        const size_type nbDof = M_levelSetIntersect [ otherFractureId ]->get_mesh_fem().nb_basic_dof();

        for ( size_type d = 0; d < nbDof; ++d )
        {
            base_node node = M_levelSetIntersect [ otherFractureId ]->get_mesh_fem().point_of_basic_dof(d);
            base_node mappedNode ( node.size() +1 );
	  scalar_type t = d*1./(M_data.getSpatialDiscretization () );
	  base_node P (node.size());
	  P [0] = t;
           mappedNode [0] = node [0];
           mappedNode [1] = M_levelSet->getData()->y_map ( P );
	  M_levelSetIntersect [ otherFractureId ]->values(0)[d] = otherLevelSet->getData()->ylevelSetFunction ( mappedNode );
        }

        M_meshLevelSetIntersect[ otherFractureId ]->add_level_set ( *M_levelSetIntersect [ otherFractureId ] );
        M_meshLevelSetIntersect[ otherFractureId ]->adapt ();

        size_type i_cv = 0;
        dal::bit_vector bc_cv = M_meshLevelSetIntersect[ otherFractureId ]->linked_mesh().convex_index();
		
        for ( i_cv << bc_cv; i_cv != size_type(-1); i_cv << bc_cv )
        {
            if ( M_meshLevelSetIntersect[ otherFractureId ]->is_convex_cut ( i_cv ) )
            {  
            	if ( type == "Cross")
            	{	
			M_meshFlat.region ( FractureHandler::FRACTURE_UNCUT * ( M_ID + 1 ) ).sup ( i_cv );
			M_meshFlat.region ( FractureHandler::FRACTURE_INTERSECT * ( M_ID + 1 ) + otherFractureId + 1 ).add( i_cv );
			M_extendedPressure.push_back ( M_meshFEMPressure.ind_basic_dof_of_element ( i_cv )[0] );
			M_extendedVelocity.push_back ( M_meshFEMVelocity.ind_basic_dof_of_element ( i_cv )[0] );
			M_extendedVelocity.push_back ( M_meshFEMVelocity.ind_basic_dof_of_element ( i_cv )[1] );
            	}
            	
		M_fractureIntersectElements [ otherFractureId ].push_back ( i_cv );
		pairSize_Type coppia;
		coppia.first = globalIndex;
		coppia.second = 0;
		M_fractureIntersectElementsGlobalIndex [ otherFractureId ].push_back ( coppia );
		++globalIndex;
		++numIntersect;
	          }
        }
    }

    return numIntersect;
}
\end{Code}

\section{La Classe \texttt{FractureData}}

\section{La Classe \texttt{LevelSetHandler}}

\section{La Classe \texttt{LevelSetData}}


\section{La Classe \texttt{FractureIntersect}}

\section{La Classe \texttt{IntersectData}}

