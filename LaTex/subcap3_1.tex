\lstnewenvironment{Code03_01}[1][]{\lstset{basicstyle=\small\ttfamily, columns=fullflexible,framexrightmargin=+.1\textwidth, keywordstyle=\color{red}\bfseries, commentstyle=\color{blue},language=C++, basicstyle=\small,  numberstyle=\tiny, stepnumber=1, numbersep=5pt, frame=shadowbox, #1}}{}


\chapter{Strumenti di sviluppo}
Lo sviluppo della nostra libreria ha richiesto l'uso di strumenti specifici atti a facilitare e sistematizzare la gestione del codice quali \texttt{git}, \texttt{CMake} e \texttt{Dxygen}.\\
\href{https://github.com/}{\texttt{Git}} è un sistema di controllo versione utilizzavile direttamente da linea di comando, molto diffuso è utile per tenere traccia delle varie fasi di sviluppo del codice. Git gestisce in modo adeguato i contributi al codice provenienti da agenti esterni e permette la condivisione del codice.\\
\href{http://www.cmake.org/}{\texttt{CMake}} è un software libero multipiattaforma nato per l'automazione dello sviluppo. Sostanzialmente con l'uso di \texttt{CMake} è possibile configurare e generare Makefile per il sistema operativo in uso. Questo facilita la diffusione del codice tra sviluppatori, non dovendo far altro che lanciare \texttt{CMake} e lasciare che sia lui ad occuparsi della ricerca di compilatori e librerie locali e della costruzione di software.\\
\href{www.doxygen.org/}{\texttt{Doxygen}} è un sistema multipiattaforma molto diffuso per la generazione automatica della documentazione di un codice. Per ottenere la documentazione è necessario introdurre nel codice dei commenti con una sintassi particolare, e il risultato è contiene l’elenco delle classi implementate, i loro diagrammi di collaborazione e la descrizione di metodi ed attributi.

\section{\texttt{Git}}
Il codice del progetto è reperibile su \texttt{git} ed è possibile scaricarlo e collabolare allo sviluppo clonando il codice dalla repository  \href{https://github.com/}{\texttt{GitHub}}:\\
\begin{center}
\texttt{ git clone https://github.com/mariaiemoli/progetto.git}
\end{center}
Nella cartella principale è contenuto anche un file \texttt{.gitignore}, in cui sono specificate le estensioni dei files e le sottocartelle che non devono essere visionati in una repository \texttt{git}. In particolare non si è interessati ai files temporanei che vengono eventualmente generati dagli editor, e alle cartelle generate da uno scorretta configurazione di \texttt{Doxygen}.

\section{\texttt{CMake}}
\texttt{CMake} si basa sulla creazione di files \texttt{CMakeLists.txt} nella directory del progetto, che contengono le direttive necessarie per creare il \texttt{Makefile} per compilare ed eseguire il codice. \\
Nella directory principale si trova un primo file \texttt{CMakeLists.txt} in cui si impostano i parametri fondamentali quali la dipendenza dalla versione di \texttt{CMake}, il nome del progetto, si includono le directory dei sorgenti nel path di compilazione e si caricano le librerie esterne utilizzate dai sorgenti. \\
le librerie che vengono caricate sono:
\begin{itemize}
\item \texttt{Eigen}, libreria template per l'algebra lineare usata principalmente nella fase di imposizione delle condizioni di interfaccia sul triangolo di intersezione nel caso della biforcazione, per il modello ridotto;
\item \texttt{Getfem++}, libreria matematica per gli elementi finiti usata per scrivere e risolvere il problema numerico;
\item \texttt{Blas} (Basic Linear Algebra Subprograms) e \texttt{Qhull}, librerie necessarie per \texttt{Getfem++};
\item \texttt{Doxygen}, usato per generare la documentazione
\end{itemize}

\begin{Code03_01}
CMAKE_MINIMUM_REQUIRED( VERSION 2.8 )

PROJECT(PACS)

SET(CMAKE_CXX_FLAGS "-std=c++0x -Wall ${CMAKE_CXX_FLAGS}")

SET(CMAKE_MODULE_PATH 
	${CMAKE_SOURCE_DIR}/cmake ${CMAKE_MODULE_PATH})

# Include Eigen3
FIND_PACKAGE(PkgConfig)
PKG_CHECK_MODULES(EIGEN3 REQUIRED eigen3)
INCLUDE_DIRECTORIES(${EIGEN3_INCLUDE_DIRS})

# Include Doxygen
find_package(Doxygen)
[ ... ]

# Include Getfem
if (GETFEM_LIBRARIES AND GETFEM_INCLUDE_DIRS)
[ ... ]

#Include for BLAS library
find_package ( BLAS )
[ ... ]


#Include for LAPACK library
find_package ( LAPACK )
[ ... ]

#Include for QHULL library
set(QHULL_MAJOR_VERSION 6)
[ ... ]

SET(CMAKE_INSTALL_PREFIX 
	${CMAKE_SOURCE_DIR}/install-dir CACHE PATH "" FORCE)

ADD_SUBDIRECTORY(src)
ADD_SUBDIRECTORY(test)
\end{Code03_01}

Per poter usare \texttt{CMake} è necessario posizionarsi nella directory in cui è contenuto il progetto e creare una cartella in cui compilare il codice. Per lanciare \texttt{CMake} è necessario digitare i seguenti comandi: \\ 
%\begin{center}
 \texttt{cd $<$build\_dir$>$ } \\
\texttt{cmake $<$source\_dir$>$}
%\end{center}

\texttt{CMake} imposta le variabili di path come definito nel file  \texttt{CMakeLists.txt}, esamina le dipendenze da librerie esterne e procede esaminando le sottocartelle aggiunte nel file di configurazione. In particolare esamina le sottocartelle \texttt{src} e \texttt{test}, in cui sono presenti altri files  \texttt{CMakeLists.txt}, che impostano i rispettivi obiettivi e definiscono le rispettive sottocartelle. Questo permette a \texttt{CMake} di esaminare tutta la directory in cui è contenuto il progetto in maniera ricorsiva. \\
Nella cartella \texttt{src} il file  \texttt{CMakeLists.txt} aggiunge la creazione di una libreria a partire dal codice del progetto. 
\begin{Code03_01}
ADD_LIBRARY(pacs ${TARGET_SRC} ${TARGET_HANDLER})
\end{Code03_01}

Nella cartella test si trovano dei files data di esempio per poter eseguire il codice, contenuti della sottocartella \texttt{data}. Il file  \texttt{CMakeLists.txt} nella cartella dei test definisce gli obiettivi e i collegamenti necessari per l'esecuzione. 

\par Una volta che il \texttt{Makefile} è stato generato è possibile compilare il codice con il comando: \\ 
\texttt{make}.
Questo crea nella sottocartella \texttt{test} un eseguibile chiamato \texttt{darcy}. Per poterlo eseguire è sufficiente lanciarlo da terminale dandogli in pasto uno dei files presenti nella sottocartella \texttt{data}:\\
\texttt{ .\textbackslash darcy data\textbackslash$<$file\_data$>$ }

\section{\texttt{Doxygen}}
Per poter creare la documentazione al codice con \texttt{Doxygen} è necessario che nella cartella principale sia presente il file \texttt{Doxygen}, in caso contrario è necessario crearlo con il comando:\\
\texttt{Doxygen -g}\\
In questo file sono definite le variabili che permettono di creare la documentazione come il nome del progetto, la directory dove creare la documentazione, le cartelle dove si trovano i sorgenti e l'estensione dei files da considerare.

\begin{Code03_01}
PROJECT_NAME 				=	 Problema di Darcy in un network di fratture
OUTPUT_DIRECTORY			=	./build/doc
INPUT						= 	./src ./test
FILE_PATTERNS				= 	*.cc *.h
HAVE_DOT					=	YES
COLLABORATION_GRAPH		=	YES
HIDE_UNDOC_RELATIONS		=	NO
\end{Code03_01}

Dopo aver lanciato \texttt{CMake} è possibile generare la documentazione semplicemente eseguendo: \\
\texttt{cd $<$build\_dir$>$} \\
\texttt{make doc}

Questo crea una sottocartella \texttt{build/doc} che contiene la documentazione \texttt{html} e \LaTeX{}.