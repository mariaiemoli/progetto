\lstnewenvironment{Code03_01}[1][]{\lstset{basicstyle=\small\ttfamily, columns=fullflexible,framexrightmargin=+.1\textwidth, keywordstyle=\color{red}\bfseries, commentstyle=\color{blue},language=C++, basicstyle=\small, numbers=left, numberstyle=\tiny, stepnumber=1, numbersep=5pt, frame=shadowbox, #1}}{}

\chapter{Strumenti di sviluppo}
Lo sviluppo della nostra libreria ha richiesto l'uso di strumenti specifici atti a facilitare e sistematizzare la gestione del codice quali \texttt{git}, \texttt{CMake} e \texttt{Dxygen}.\\
\href{https://github.com/}{\texttt{Git}} è un sistema di controllo versione utilizzavile direttamente da linea di comando, molto diffuso è utile per tenere traccia delle varie fasi di sviluppo del codice. Git gestisce in modo adeguato i contributi al codice provenienti da agenti esterni e permette la condivisione del codice.\\
\href{http://www.cmake.org/}{\texttt{CMake}} è un software libero multipiattaforma nato per l'automazione dello sviluppo. Sostanzialmente con l'uso di \texttt{CMake} è possibile configurare e generare Makefile per il sistema operativo in uso. Questo facilita la diffusione del codice tra sviluppatori, non dovendo far altro che lanciare \texttt{CMake} e lasciare che sia lui ad occuparsi della ricerca di compilatori e librerie locali e della costruzione di software.\\
\href{www.doxygen.org/}{\texttt{Doxygen}} è un sistema multipiattaforma molto diffuso per la generazione automatica della documentazione di un codice. Per ottenere la documentazione è necessario introdurre nel codice dei commenti con una sintassi particolare, e il risultato è contiene l’elenco delle classi implementate, i loro diagrammi di collaborazione e la descrizione di metodi ed attributi.

\section{\texttt{Git}}
Il codice del progetto è reperibile su \texttt{git} ed è possibile scaricarlo e collabolare allo sviluppo clonando il codice dalla repository  \href{https://github.com/}{\texttt{GitHub}}:\\
\begin{center}
\texttt{ git clone https://github.com/mariaiemoli/progetto.git}
\end{center}
Nella cartella principale è contenuto anche un file \texttt{.gitignore}, in cui sono specificate le estensioni dei files e le sottocartelle che non devono essere visionati in una repository \texttt{git}. In particolare non si è interessati ai files temporanei che vengono eventualmente generati dagli editor, e alle cartelle generate da uno scorretta configurazione di \texttt{Doxygen}.

\section{\texttt{CMake}}


\section{\texttt{Doxygen}}