\chapter{Risultati numerici e validazione del modello} 
Per validare il modello ridotto da noi utilizzato abbiamo confrontato i nostri risultati con quelli ottenti con dei codici \texttt{FreeFem++}, in cui il problema è bidimensionale. Per ogni caso preso in considerazione sono state effettuate tre simulazioni, con spessori che ad ogni passo diminuiscono di un ordine di grandezza. Quello che ci aspettiamo infatti è che i risultati ottenuti con \texttt{FreeFem++}  si avvvicinino ai nostri risultati diminuendo lo spessore delle fratture.\\
\noindent Per ogni frattura abbiamo messo a confronto il valore della pressione nel DOF di intersezione, per il modello monodimensionale, con l'integrale di pressione lungo il lato del triangolo di intersezione mediato sulla sua lunghezza, per \texttt{FreeFem++}.
Riportiamo inoltre il valore medio della pressione nel punto di intersezione ponendolo a paragone con il valore dell'integrale di pressione sul triangolo di intersezione mediato per l'area di quest'ultimo.\\
Vediamo ora i risultati ottenuti in alcuni casi significativi.
\section{Biforcazione semplice, 3 fratture}
Consideriamo una biforcazione generata dall'incontro di tre fratture in uno dei loro punti estremi.
Confronto valori della pressione nel DOF di intersezione per il modello 2D:\\
\begin{center}
\begin{tabular}{|c|c|c|c|c|}
\hline
 & \textbf{\texttt{Code}} & \multicolumn{3}{|c|}{\textbf{\texttt{FreeFem++}}} \\
\hline
\multicolumn{1}{|c|}{Frattura} & - &
\multicolumn{1}{|c|}{1E-01} & 1E-02 & 1E-03 \\
\hline
 0 & 0.000594764 & 0.372412 & 0.0674785 & -0.000186152\\
 1 & -0.0013522 & 0.244632 & 0.0553189 & -0.000189784\\
 2 & 0.00225249 & 0.323944 & 0.0627701 & -0.000187243\\
\hline
\end{tabular}
\end{center}

\begin{center}
\begin{tabular}{|c|c|c|}
\hline
 & \textbf{\texttt{Code}} & \multicolumn{1}{|c|}{\textbf{\texttt{FreeFem++}}} \\
\hline
 Pressione media & 0.0004983513 & -0.000187683\\
\hline
\end{tabular}
\end{center}

\begin{figure}[h!]
\centering
\includegraphics[scale=.4]{img/cap6/Biforcazione2D.eps}
\caption{Risultati \texttt{FreeFem++} biforcazione - spessore ordine 1E-02 }\label{Biforcazione1E-02}
\end{figure}

\section{Doppia biforcazione verticale, 5 fratture}
Vediamo ora i valori in un network contenente due biforcazioni con una frattura in comune.
La prima biforcazione coinvolge le fratture 0,1,3 mentre la seconda nasce dall'incontro fra 2,3,4.\\

\begin{figure}[h!]
\centering
\includegraphics[scale=.45]{img/cap6/DoppiaBifu.eps}
\caption{Risultati \texttt{FreeFem++} doppia biforcazione risultati paraview }\label{DoppiaBifuVerticalParaview}
\end{figure}

Per la frattura 3 troveremo dapprima il valore relativo al lato del triangolo di intersezione 0-1-3, e subito dopo il valore per quello realativo alla biforcazione 2-3-4.

\begin{center}
\begin{tabular}{|c|c|c|c|c|}
\hline
\multicolumn{5}{|c|}{Biforcazione 0-1-3}\\
\hline
 & \textbf{\texttt{Code}} & \multicolumn{3}{|c|}{\textbf{\texttt{FreeFem++}}} \\
\hline
\multicolumn{1}{|c|}{Frattura} & - &
\multicolumn{1}{|c|}{1E-01} & 1E-02 & 1E-03 \\
\hline
 0 & 0.000463907 & 0.0045642 & -0.112138 & -0.0198579\\
 1 & -0.00101445 & 0.0608937 & -0.107373 & -0.0196711\\  
 3 & 0.00252798 & 0.063068 & -0.109636 & -0.0197122\\
\hline
\multicolumn{5}{|c|}{Biforcazione 2-3-4}\\
\hline
 & \textbf{\texttt{Code}} & \multicolumn{3}{|c|}{\textbf{\texttt{FreeFem++}}} \\
\hline
\multicolumn{1}{|c|}{Frattura} & - &
\multicolumn{1}{|c|}{1E-01} & 1E-02 & 1E-03 \\
\hline
 2 & 0.00409494 & 0.783991 & 0.477659 & 0.0278173\\ 
 3 & -0.00956599 & 0.83018 & 0.485314 & 0.0282011\\
 4 & 0.00526259 & 0.809903 & 0.480369 & 0.02788\\
\hline
\end{tabular}
\end{center}

\begin{figure}[h!]
\centering
\includegraphics[scale=.35]{img/cap6/DoppiaBifuVertical.eps}
\caption{Risultati \texttt{FreeFem++} doppia biforcazione - spessore ordine 1E-02 }\label{DoppiaBifuVertical1E-02}
\end{figure}

\begin{center}
\begin{tabular}{|c|c|c|c|c|}
\hline
 Pressione media & \textbf{\texttt{Code}} & \multicolumn{3}{|c|}{\textbf{\texttt{FreeFem++}}} \\ 
\hline
\multicolumn{1}{|c|}{Biforcazione} & - &
\multicolumn{1}{|c|}{1E-01} & 1E-02 & 1E-03 \\
\hline
  0-1-3 & 6.59*1E-04 & 0.0393046 & -0.109636 & -0.0197574 \\
  2-3-4 & -6.94*1E-05 & 0.810176 & 0.481358 & 0.0279764 \\
\hline
\end{tabular}
\end{center}

\begin{center}
\begin{tabular}{|c|c|c|c|}
\hline
  \multicolumn{3}{|c|}{\textbf{\texttt{Errore}}} \\ 
\hline
\multicolumn{1}{|c|}{Biforcazione} & - &
\multicolumn{1}{|c|}{1E-01} & 1E-02 & 1E-03 \\
\hline
  0-1-3 &  0.0386456 & 0.110295 & 0.0204164 \\
  2-3-4 & 0.8102454 & 0.4844274 & 0.0280458 \\
\hline
\end{tabular}
\end{center}


\section{Doppia biforcazione e cross, 6 fratture}
Andiamo ora ad analizzare i valori in un network contenente due biforcazioni e un cross generati da un insieme di 6 fratture. In particolare le fratture con intersezione di tipo \textit{Cross} sono 2-4. Mentre quelle con intersezione di tipo \textit{bifurcation} sono 0-1-2 e 3-4-5;\\

\begin{figure}[h!]
\centering
\includegraphics[scale=.5]{img/cap6/DoppiaBifuCross.eps}
\caption{Risultati \texttt{FreeFem++} doppia biforcazione e cross risultati paraview }\label{DoppiaBifuCrossParaview}
\end{figure}

\begin{center}
\begin{tabular}{|c|c|c|c|c|}
\hline
\multicolumn{5}{|c|}{Biforcazione 0-1-3}\\
\hline
 & \textbf{\texttt{Code}} & \multicolumn{3}{|c|}{\textbf{\texttt{FreeFem++}}} \\
\hline
\multicolumn{1}{|c|}{Frattura} & - &
\multicolumn{1}{|c|}{1E-01} & 1E-02 & 1E-03 \\
\hline
 0 & -0.00402293 & -0.0126464 & -8.90379e-05 & 4.71909e-08\\
 1 & -0.0113048 & 0.00217558 & -7.84109e-05 & -9.00287e-07\\  
 2 & -0.00999955 & 0.0123638 & 3.64042e-06 & 5.78993e-07\\
\hline
\multicolumn{5}{|c|}{Biforcazione 3-4-5}\\
\hline
 & \textbf{\texttt{Code}} & \multicolumn{3}{|c|}{\textbf{\texttt{FreeFem++}}} \\
\hline
\multicolumn{1}{|c|}{Frattura} & - &
\multicolumn{1}{|c|}{1E-01} & 1E-02 & 1E-03 \\
\hline
 3 & -0.0319253 & 0.0917434 & 2.15017e-05 & -2.49308e-06\\ 
 4 & -0.129783 & 0.135803 & 4.0555e-05 & 1.01509e-05\\
 5 & -0.010618 & 0.154797 & 1.12678e-05 & -1.04707e-05\\
\hline
%\multicolumn{5}{|c|}{Cross 2-4}\\
%\hline
% & \textbf{\texttt{Code}} & \multicolumn{3}{|c|}{\textbf{\texttt{FreeFem++}}} \\
%\hline
%\multicolumn{1}{|c|}{Frattura} & - &
%\multicolumn{1}{|c|}{1E-01} & 1E-02 & 1E-03 \\
%\hline
% 2 & -0.00999955 & 0.803246 & 0.522995 & 0.0154489\\ 
% 4 & -0.129783 & 0.778237 & 0.520145 & 0.0154059\\
%\hline
\end{tabular}
\end{center}

\begin{center}
\begin{tabular}{|c|c|c|c|c|}
\hline
 Pressione media & \textbf{\texttt{Code}} & \multicolumn{3}{|c|}{\textbf{\texttt{FreeFem++}}} \\ 
\hline
\multicolumn{1}{|c|}{Biforcazione} & - &
\multicolumn{1}{|c|}{1E-01} & 1E-02 & 1E-03 \\
\hline
  0-1-2 & - & 0.00589771 & 1.76686e-06 & 1.37214e-07 \\
  3-4-5 & - & 0.127337 & -2.32565e-05 & -5.81148e-07 \\
\hline
\end{tabular}
\end{center}
