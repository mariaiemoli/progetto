\chapter{Risultati numerici e validazione del modello} 
Per validare il modello ridotto utilizzato abbiamo confrontato i nostri risultati con quelli ottenuti con dei codici \texttt{FreeFem++}, in cui è stato implementato lo stesso problema ma bidimensionale. Per ogni caso preso in considerazione sono state effettuate tre simulazioni, con spessori che ad ogni passo diminuiscono di un ordine di grandezza. Quello che ci aspettiamo infatti \`{e} che i risultati ottenuti con \texttt{FreeFem++}  si avvicinino ai nostri diminuendo lo spessore delle fratture.\\
\noindent Per ogni frattura abbiamo messo a confronto il valore della pressione nel DOF di intersezione, per il modello monodimensionale, con l'integrale di pressione lungo il lato del triangolo di intersezione mediato sulla sua lunghezza, per \texttt{FreeFem++}.
Riportiamo inoltre il valore medio della pressione nel punto di intersezione confrontandolo con il valore dell'integrale di pressione sul triangolo di intersezione mediato per l'area di quest'ultimo.\\
Vediamo ora i risultati ottenuti in alcuni casi significativi.
\section{Biforcazione semplice, tre fratture}
Consideriamo una biforcazione generata dall'incontro di tre fratture in uno dei loro punti estremi.
Confronto valori della pressione nel DOF di intersezione per il modello 2D:\\
\begin{center}
\begin{tabular}{|c|c|c|c|c|}
\hline
 & \textbf{\texttt{Code}} & \multicolumn{3}{|c|}{\textbf{\texttt{FreeFem++}}} \\
\hline
\multicolumn{1}{|c|}{Frattura} & - &
\multicolumn{1}{|c|}{1E-01} & 1E-02 & 1E-03 \\
\hline
 0 & 0.000594764 & -0.476811 & -0.127327 & -6.44886e-06\\
 1 & -0.0013522 & -0.521943 & -0.130901 & -7.2092e-06\\
 2 & 0.00225249 & -0.47029 & -0.126459 & -6.55509e-06\\
\hline
\end{tabular}
\end{center}

\begin{center}
\begin{tabular}{|c|c|c|c|}
\hline
\multicolumn{4}{|c|}{Pressione media} \\
\hline
\textbf{\texttt{Code}} & \multicolumn{3}{|c|}{\textbf{\texttt{FreeFem++}}} \\
\hline
- & \multicolumn{1}{|c|}{1E-01} & 1E-02 & 1E-03 \\
\hline
0.000498351 & -0.49058 & -0.127993 & -6.72695e-06 \\
\hline
\end{tabular}
\end{center}

\begin{figure}[h!]
\centering
\includegraphics[scale=.4]{img/cap6/Biforcazione2D.eps}
\caption{Risultati \texttt{FreeFem++} biforcazione - spessore ordine 1E-02 }\label{Biforcazione1E-02}
\end{figure}

\begin{center}
\begin{tabular}{|c|c|c|}
\hline
  \multicolumn{3}{|c|}{\textbf{\texttt{Errore}}} \\ 
\hline
\multicolumn{1}{|c|}{1E-01} & 1E-02 & 1E-03 \\
\hline
 0.490081649 & 0.127494649 & 0.000491624 \\
\hline
\end{tabular}
\end{center}

\section{Doppia biforcazione verticale, cinque fratture}
Vediamo ora i valori in un network contenente due biforcazioni con una frattura in comune.
La prima biforcazione coinvolge le fratture \textit{0,1,3} mentre la seconda nasce dall'incontro fra \textit{2,3,4}.

\begin{figure}[h!]
\centering
\includegraphics[scale=.45]{img/cap6/DoppiaBifuVerticalF0.eps}
\caption{Risultati \texttt{FreeFem++} doppia biforcazione risultati paraview }\label{DoppiaBifuVerticalParaview}
\end{figure}

\noindent Per la frattura \textit{3} troveremo dapprima il valore relativo al lato del triangolo di intersezione \textit{0-1-3}, e subito dopo il valore per quello realativo alla biforcazione \textit{2-3-4}.

\begin{center}
\begin{tabular}{|c|c|c|c|c|}
\hline
\multicolumn{5}{|c|}{Biforcazione 0-1-3}\\
\hline
 & \textbf{\texttt{Code}} & \multicolumn{3}{|c|}{\textbf{\texttt{FreeFem++}}} \\
\hline
\multicolumn{1}{|c|}{Frattura} & - &
\multicolumn{1}{|c|}{1E-01} & 1E-02 & 1E-03 \\
\hline
 0 & 0.000594764 & 0.0045642 & -0.112138 & -0.0198579\\
 1 & -0.0013522 & 0.0608937 & -0.107373 & -0.0196711\\  
 3 & 0.00225249 & 0.063068 & -0.109636 & -0.0197122\\
\hline
\multicolumn{5}{|c|}{Biforcazione 2-3-4}\\
\hline
 & \textbf{\texttt{Code}} & \multicolumn{3}{|c|}{\textbf{\texttt{FreeFem++}}} \\
\hline
\multicolumn{1}{|c|}{Frattura} & - &
\multicolumn{1}{|c|}{1E-01} & 1E-02 & 1E-03 \\
\hline
 2 & 0.00409494 & 0.783991 & 0.477659 & 0.0278173\\ 
 3 & -0.00956599 & 0.83018 & 0.485314 & 0.0282011\\
 4 & 0.00526259 & 0.809903 & 0.480369 & 0.02788\\
\hline
\end{tabular}
\end{center}

\begin{center}
\begin{tabular}{|c|c|c|c|c|}
\hline
 Pressione media & \textbf{\texttt{Code}} & \multicolumn{3}{|c|}{\textbf{\texttt{FreeFem++}}} \\ 
\hline
\multicolumn{1}{|c|}{Biforcazione} & - &
\multicolumn{1}{|c|}{1E-01} & 1E-02 & 1E-03 \\
\hline
  0-1-3 & 6.59e-04 & 0.0393046 & -0.109636 & -0.0197574 \\
  2-3-4 & -6.94e-05 & 0.810176 & 0.481358 & 0.0279764 \\
\hline
\end{tabular}
\end{center}

\begin{figure}[h!]
\centering
\includegraphics[scale=.35]{img/cap6/DoppiaBifuVertical.eps}
\caption{Risultati \texttt{FreeFem++} doppia biforcazione - spessore ordine 1E-02 }\label{DoppiaBifuVertical1E-02}
\end{figure}

\begin{center}
\begin{tabular}{|c|c|c|c|}
\hline
  \multicolumn{4}{|c|}{\textbf{\texttt{Errore}}} \\ 
\hline
\multicolumn{1}{|c|}{Biforcazione} &
\multicolumn{1}{|c|}{1E-01} & 1E-02 & 1E-03 \\
\hline
  0-1-3 &  0.0386456 & 0.110295 & 0.0204164 \\
  2-3-4 & 0.8102454 & 0.4844274 & 0.0280458 \\
\hline
\end{tabular}
\end{center}


\section{Doppia biforcazione e cross, 6 fratture}
Andiamo ora ad analizzare i valori in un network contenente due biforcazioni e un cross generati da un insieme di sei fratture. In particolare le fratture con intersezione di tipo \textit{Cross} sono \textit{2-4}. Mentre quelle con intersezione di tipo \textit{bifurcation} sono \textit{0-1-2} e \textit{3-4-5}.\\

\begin{figure}[h!]
\centering
\includegraphics[scale=.5]{img/cap6/DoppiaBifuCrossF0.eps}
\caption{Risultati \texttt{FreeFem++} doppia biforcazione e cross risultati paraview }\label{DoppiaBifuCrossParaview}
\end{figure}

\begin{center}
\begin{tabular}{|c|c|c|c|c|}
\hline
\multicolumn{5}{|c|}{Biforcazione 0-1-3}\\
\hline
 & \textbf{\texttt{Code}} & \multicolumn{3}{|c|}{\textbf{\texttt{FreeFem++}}} \\
\hline
\multicolumn{1}{|c|}{Frattura} & - &
\multicolumn{1}{|c|}{1E-01} & 1E-02 & 1E-03 \\
\hline
 0 & -8.21235e-07 & -0.0105826 & 4.86992e-05 & -7.46571e-07\\
 1 & -0.00018082 & 0.00114913 & 4.70438e-05 & 8.4115e-08\\  
 2 & -0.00350292 & 0.0152323 & -2.22648e-05 & -3.94552e-06\\
\hline
\multicolumn{5}{|c|}{Biforcazione 3-4-5}\\
\hline
 & \textbf{\texttt{Code}} & \multicolumn{3}{|c|}{\textbf{\texttt{FreeFem++}}} \\
\hline
\multicolumn{1}{|c|}{Frattura} & - &
\multicolumn{1}{|c|}{1E-01} & 1E-02 & 1E-03 \\
\hline
 3 & -0.00617284 & 0.0785465 & 4.84694e-06 & 5.45724e-07\\ 
 4 & -0.0257199 & 0.118724 & 7.17945e-05 & 1.8384e-05\\
 5 & -0.00203255 & 0.135597 & 3.95891e-05 & -8.71332e-06\\
\hline
\end{tabular}
\end{center}

\begin{center}
\begin{tabular}{|c|c|c|c|c|}
\hline
 Pressione media & \textbf{\texttt{Code}} & \multicolumn{3}{|c|}{\textbf{\texttt{FreeFem++}}} \\ 
\hline
\multicolumn{1}{|c|}{Biforcazione} & - &
\multicolumn{1}{|c|}{1E-01} & 1E-02 & 1E-03 \\
\hline
  0-1-2 & -0.001228187 & 0.00725767 & -2.96401e-06 & 7.36626e-07 \\
  3-4-5 & -0.01130843 & 0.109662 & -3.78985e-05 & -6.11941e-07 \\
\hline
\end{tabular}
\end{center}

\begin{center}
\begin{figure}[h!]
\centering
\includegraphics[scale=.075]{img/cap6/Doppiabifucross3.eps}
\caption{Risultati \texttt{FreeFem++} doppia biforcazione con cross - spessore ordine 1E-02 }\label{DoppiaBifuCross1E-02}
\end{figure}
\end{center}

\begin{center}
\begin{tabular}{|c|c|c|c|}
\hline
  \multicolumn{4}{|c|}{\textbf{\texttt{Errore}}} \\ 
\hline
\multicolumn{1}{|c|}{Biforcazione} &
\multicolumn{1}{|c|}{1E-01} & 1E-02 & 1E-03 \\
\hline
  0-1-2 &  0.006029483 & 0.001225223 & 0.00122745 \\
  3-4-5 & 0.09835357 & 0.011270532 & 0.011307818 \\
\hline
\end{tabular}
\end{center}