\lstnewenvironment{Code03_03}[1][]{\lstset{basicstyle=\small\ttfamily, columns=fullflexible,framexrightmargin=+.1\textwidth, keywordstyle=\color{red}\bfseries, commentstyle=\color{blue},language=C++, basicstyle=\small, numbers=left, numberstyle=\tiny, stepnumber=1, numbersep=5pt, frame=shadowbox, #1}}{}

\chapter{Classi per la gestione del problema numerico}

\section{Class \texttt{DarcyFracture}}
La classe \texttt{DarcyFracture} ci permette di passare alla formulazione algebrica, $ Ax=b $, del problema e di risolverlo. \\ 

%Il suo costruttore richiede i seguenti parametri:
%	\begin{itemize}
%	\item \textit{Mesh:} relativa al mezzo o a una delle fratture
%	\item \textit{Stringa:} contenente il tipo di mesh di GetFem del elemento precedente
%	\item \textit{Vettore:} nullo, se la mesh fa riferimento al mezzo, contenete gli indici dei gradi di libert\`{a} degli estremi, se si sta lavorando su una frattura
%	\item \textit{ElementDimesion:} dimensione in cui stiamo lavorando ( di defaul 2 per il mezzo e 1 per le fratture)
%	\end{itemize}
I campi fondamentali della classe sono:
	\begin{enumerate}
	\item[-] \texttt{M\_globalMatrix}: \`{e} una matrice a blocchi sparsa rappresentata i GETfem da un \textit{boost::shared\_ptr $\langle$ gmm::row\_matrix $\langle$ sparseVector\_Type $\rangle \rangle$};
	\item[-] \texttt{M\_globalRightHandSide}: vettore del termine noto;
	\item[-] \texttt{M\_velocityAndPressure}: vettore soluzione delle velocit\'{a} e della pressione.
	\end{enumerate} 

\begin{Code03_03}[caption={Classe \texttt{DarcyFracture}}]
public:

    DarcyFractured ( const MediumDataPtr_Type& medium,
                     const MeshHandlerPtr_Type& mesh,
                     const BCHandlerPtr_Type& bcHandler,
                     const FracturesSetPtr_Type& fractures,
                     const ExporterPtr_Type& exporter );
    
    void init ( );
    
    void assembly ( const GetPot& dataFile );

    void solve ( );

	[ ... ]

private:

	[ ... ]

    // Global matrix, darcy
    sparseMatrixPtr_Type M_globalMatrix;
    // Termine noto di destra del sistema
    scalarVectorPtr_Type M_globalRightHandSide;
    // Soluzione del sistema ( velocita' + pressione )
    scalarVectorPtr_Type M_velocityAndPressure;

    [ ... ]

};
\end{Code03_03}

Il metodo init() va a settare tutte le costanti note del problema.
Il metodo assembly() va a creare la matrice globale $A$ e il termine noto di destra, utilizzando le funzioni della classe \texttt{XFEMOperators}. 
Ipotizzando di avere un ambiente dove tre fratture hanno un intersezione di tipo \textit{Bifurcation} e una di queste si incontra con una quarta creando un intersezione di tipo \textit{Cross}, la matrice globale avrà il seguente volto:\\
%---->DA SISTEMAREEEEE <----------------- 
 \begin{center}
  $ \left[ \begin{matrix}
 			A_{0} &  B_{0} & 0 & 0 & 0 & 0 & 0 & 0 \\ 
 			B_{0}^{T} & 0 & 0 & 0 & 0 & 0 & 0 & 0 \\
 			0 & 0 & A_{1} &  B_{1} & 0 & 0 & 0 & 0 \\ 
		 	0 & 0 & B_{1}^{T} & 0 & 0 & 0 & 0 & 0 \\
		 	0 & 0 & 0 & 0 & A_{2} &  B_{2} & 0 & 0 \\ 
		 	0 & 0 & 0 & 0 & B_{2}^{T} & 0 & 0 & 0 \\
		 	0 & 0 & 0 & 0 & 0 & 0 & A_{3} &  B_{3} \\ 
 			0 & 0 & 0 & 0 & 0 & 0 & B_{3}^{T} & 0 \\
 			\end{matrix}.\right] $ 
  \end{center}
Nel caso di un \textit{Cross} la matrice App viene generata durante l'assembly() per garantire che le pressioni relative ai gradi di libert\`{a} estesi siano uguali nel punto di intersezione. Mentre le matrici $E_{i}$ sono inserite per imporre le condizioni di interfaccia.\\
Nel caso di una \textit{Bifurcation} per trovare le condizioni da rispettare nel punto di intersezione dobbiamo porci in uno spazio 2D.
Le fratture avranno dunque uno spessore e potremo delimitare un triangolo di intersezione unendo i punti di incontro dei loro bordi. \\
Presi \textbf{N} il vettore delle normali ai lati del triangolo, \textbf{$K_{I}$} la matrice di permeabilit\`{a} all'intersezione e una costante \textit{t} possiamo definire:
\begin{center}
	$ S = \textit{t}diag( \textbf{N}$ \textbf{$K_{I}$} $ \textbf{N}^{T} ) $
\end{center}
Definite ora \textbf{C} matrice dei vettori che uniscono il baricentro del triangolo con il punto medio  di ogni lato e \textbf{$P_{c}$} matrice di proiezione sullo spazio nullo della matrice di base per \textbf{C}, possiamo trovare la matrice di trasmissibilit\`{a}:
\begin{center}
	$ T = \frac{1}{\textbf{I}}( \textbf{N}$ \textbf{$K_{I}$} $ \textbf{N}^{T} ) + $\textbf{$P_{c}$} $\textbf{S}$\textbf{$P_{c}$}
\end{center}
%\textbf{u} = \left\lbrace \u_{0}, \u_{1}, \u_{2} \right\rbrace
%$\textbf{\pi} = \left\lbrace \pi_{0}, \pi_{1}, \pi_{2} \right\rbrace $
Le variabili con cui ci troviamo a lavorare sono:
	\begin{enumerate}
	\item[-] \textbf{u} : vettore dei flussi
	\item[-] $p_{I}$ :  pressione nel punto di intersezione delle assi delle tre fratture;
	\item[-] \textbf{$\pi$}: vettore delle pressioni nel punto di incontro fra l'asse della frattura i-esima con un lato del triangolo;
	\end{enumerate} 
e le condizioni che dobbiamo porre all'intersezione sono:
\begin{center}			
	$\left \{
		\begin{array}{l}	
	 		\textbf{u} - p_{I}T\textbf{1}_{3}+T\Pi=0  \\ \\
     	 	\sum_{k=0}^2 u_{k} = p_{I} - \pi  \\
		\end{array}
	\right.$
\end{center}
Dunque nel caso di una \textit{Bifurcation} prima di terminare la funzione assembly() va a imporre queste equazioni nella riga corrispondente al DOF di pressione di intersezione della frattura.\\
Infine il metodo solve() risolve il problema e esporta i risultati ottenuti in formato \emph{vtk}, sia per la pressione che per la velocit\`{a}.\\

\section{Class \texttt{XFEMOperetors}}
Ogni frattura ha associata la matrice:
\begin{center} 
 $ \left[ \begin{matrix}
 	A11 &  A12 \\ 
 	A12^{T} & 0
 \end{matrix}\right] $
\end{center} 

La classe \texttt{XFEMOperators} implementa le funzioni che ci permettono di calcolare questi blocchi.
Le sue funzioni principali sono le seguenti:

\begin{Code03_03}[caption={Funzioni per assemblare la matrice globale}]
//Costruisce  A11 = Aij = a(\phi_j, \phi_i)
void darcy_A11F ( sparseMatrixPtr_Type& M,
                  const FractureHandlerPtr_Type& fracture,
                  const scalar_type& gammaU,
                  const scalarVector_Type& invKTangentialInterpolated,
                  const sizeVector_Type &ExtBoundary,
                  const size_type& uncutRegionFlag );

// Aggiorna A11 nel caso intersezione" Cross "
void darcy_A11F_Cross ( sparseMatrixPtr_Type& M,
					    const FractureHandlerPtr_Type& fracture,
					    const scalarVector_Type& invKTangentialInterpolated,
					    const FractureHandlerPtr_Type& otherFracture,
					    const size_type& cutRegionFlag );

//Costruisce A12 = Bij = b(\phi_j, \omega_i)
void darcy_A12F ( sparseMatrixPtr_Type& M,
                  const FractureHandlerPtr_Type& fracture,
                  const size_type& uncutRegionFlag );

//Aggiorna A12 = Bij = b(\phi_j, \omega_i) nel caso di intersezione " Cross "
void darcy_A12F_Cross ( sparseMatrixPtr_Type& M,
                  	    const FractureHandlerPtr_Type& fracture,
                  	    const FractureHandlerPtr_Type& otherFracture,
                  	    const size_type& cutRegionFlag );
\end{Code03_03}

Altre importanti funzioni sono quelle che ci permetto di costruire il termine noto e son sempre contenute in questa classe.

\section{Class \texttt{PointData} e Class \texttt{TriangleData}}

Vengono utilizzate le classe \texttt{Matrix} e \texttt{Vector} della libreria \texttt{Eigen}.
Ci poniamo ora in uno spazio 2D.\\
La classe \texttt{PointData} ci permette di definire un punto e di poterlo manipolare con le operazioni di somma, differenza e prodotto.\\
La classe \texttt{TriangleData} ci permette di definire un triangolo e di calcolarne l'area. Inoltre sono stati implementati i metodi necessari per la costruzione del triangolo di intersezione.

\section{Class \texttt{Intersection}}

La class \texttt{Intersection}, definita in \texttt{TriangleHandler.h}, contiene tutte le informazioni relative ad una biforcazione in uno spazio 2D.
Le informazioni contenute nei suoi campi sono necessarie per il calcolo delle matrici associate al triangolo di intersezione.
I campi fondamentali della classe sono:
	\begin{enumerate}
	\item[-] \texttt{M\_intersection}: di tipo PointData, rappresenta il punto di intersezione;
	\item[-] \texttt{M\_tangents}: vettore contenente le tangenti alle fratture coinvolte;
	\item[-] \texttt{M\_normals}: vettore contenente le normali alle fratture coinvolte;
	\item[-] \texttt{M\_intersectionTriangle}:di tipo TriangleData, rappresenta il triangolo di intersezione, i cui vertici sono i punti d'incontro dei bordi delle fratture.
	\end{enumerate} 

\begin{Code03_03}[caption={Classe \texttt{Intersection}}]
public:
	[ ... ]
	
	/* Funzione che costruisce il triangolo di intersezione 
	 * partendo dal vettore delle fratture che si intersecano 
	 */
	void setIntersection( FracturePtrContainer_Type& M_FracturesSet );
	
	[ ... ]
private:
	
	[ ... ]	
	
	// Punto di intersezione
	PointData M_intersection;
	
	Vector2d M_tangents [ 3 ];
	Vector2d M_normals [ 3 ];
	
	// Triangolo di intersezione
	TriangleData M_intersectionTriangle;
	
\end{Code03_03}

\section{Class \texttt{MatrixBifurcationHandler}}
La class \texttt{MatrixBifurcationHandler} contiene le matrici associate a una biforcazione e implementa i metodi per calcolarle. 
%Inclusa la matrice $T$, di vitale importanza per l'imposizione delle condizioni di interfaccia del nostro problema.
I campi fondamentali della classe sono:
	\begin{enumerate}
	\item[-] \texttt{M\_intersection}: di tipo Intersection, contiene tutte le informazioni base per poter ricavare le matrici;
	\item[-] \texttt{M\_T}: matrice T di vitale importanza per l'imposizione delle condizioni di interfaccia del nostro problema.
	\end{enumerate} 
\begin{Code03_03}[caption={Classe \texttt{Intersection}}]
	/* Costruttore vuoto di defult. 
	 * Definisce la matrice 3x3 M_Pc costante pari a 1/3 
	 * riempie la matrice 2x2 M_K, leggendo dal file "data".
	 */ 
	MatrixBifurcationHandler( const GetPot& dataFile,
				const std::string& section = "mediumData/",
				const std::string& subsection = "darcy/");
	
	
	/* Funzione che inizializza M_intersection e calcola la matrice M_T 
	 */
	void setMatrices ( FracturePtrContainer_Type& fractures );
	
	[ ... ]
	
	/* Funzione che calcola la matrice M_T, matrice di trasmissibilita'. 
	 * La matrice M_T deriva dall'applicazione di uno schema alle 
	 * differenze finite mimetiche sul triangolo di intersezione per la legge di Darcy, 
	 * ignorando l'effetto della forza gravitazionale.
	 * Questa funzione richiama il calcolo delle matrici M_C, M_N e M_Qc.
	 * L'approssimazione di M_T assume la seguente forma:
	 * 					 T= N K N^T + t P_c \operatorname{trace}(NKN^T)P_c
	 */	
	void computeT( scalar_type t=6.0 );

	[ ... ]

private:

	Matrix2d M_K;
	Intersection_Type M_intersection;
	Matrix32 M_N;
	Matrix32 M_C;
	Matrix32 M_Qc;
	Matrix3d M_Pc;
	Matrix3d M_T;
	
\end{Code03_03}

\section{Class \texttt{BC} e Class \texttt{BCHandler}}
%\subsection{Definizione della classe e costruttore}
La classe \texttt{BC} ci permette di introdurre le condizioni al bordo sia nel mezzo che per ogni singola frattura. \\ 
Il campo fondamentale della classe \`{e}:
%e crea un elemento \texttt{BC} formato da:
	\begin{enumerate}
	\item[-] \texttt{M\_extBoundary}: vettore contenente l'indice dei gradi di libert\'{a} in cui sono imposte le condizioni al bordo
	\end{enumerate} 

%\newpage	
\begin{Code03_03}[caption={Classe \texttt{BC}}]
public:

    enum
    {
        DIRICHLET_BOUNDARY_NUM = 40,
        NEUMANN_BOUNDARY_NUM = 50
    };

    // Costruttore
    BC ( getfem::mesh& mesh,
         const std::string& MeshType,
         const sizeVector_Type DOFs,
         const ElementDimension& dimension = MEDIUM );

private:

    getfem::mesh_fem M_meshFEM;
    
    // flags for BC
    sizeVector_Type M_dirichlet;
    sizeVector_Type M_neumann;
    sizeVector_Type M_extBoundary;
\end{Code03_03}

Quando stiamo considerando una frattura con solo intersezioni di tipo cross, il costruttore andrà a inserire gli indici dei gradi di libert\`{a} nei rispettivi vettori valutando solamente se essi si trovano su una faccia della mesh.\\
Nel caso della biforcazione, invece, bisogna accertarsi di imporre le condizioni al bordo solamente nel' estremo libero, ovvero quello in cui non v\`{e} l'intersezione delle tre fratture.
%
%\subsection{Novit\`{a} introdotte per la biforcazione}
Per far questo quando andiamo ad analizzare una frattura \`{e} necessario fornire al costruttore il vettore dei DOF liberi.

La class \texttt{BCHandler} va ad inserire le condizioni al bordo di ogni frattura.
%Per far questo creiamo un variabile booleana che varr\`{a} 1 se il grado di libert\`{a} che sto considerando è contenuto nel vettore dei DOF liberi, 0 altrimenti. Questo mi permette di inserire un controllo prima di aggiungere quel indice ai campi \texttt{M\_dirichlet} , \texttt{M\_neumann} , \texttt{M\_extBoundary} della classe.

%\section{Class \texttt{BCHandler}}
%???????