\chapter{Classi per la gestione del problema numerico}
\section{Class BC}

\lstnewenvironment{Code}[1][]{\lstset{basicstyle=\small\ttfamily, columns=fullflexible,
keywordstyle=\color{red}\bfseries, commentstyle=\color{blue},
language=C++, basicstyle=\small,
numbers=left, numberstyle=\tiny,
stepnumber=2, numbersep=5pt, frame=shadowbox, #1}}{}

\begin{Code}[caption={Classe BC}]
public:

    enum
    {
        DIRICHLET_BOUNDARY_NUM = 40,
        NEUMANN_BOUNDARY_NUM = 50
    };

    // Costruttore
    BC ( getfem::mesh& mesh,
         const std::string& MeshType,
		 const sizeVector_Type DOFs,
         const ElementDimension& dimension = MEDIUM );

private:

    getfem::mesh_fem M_meshFEM;

    // flags for BC
    sizeVector_Type M_dirichlet;
    sizeVector_Type M_neumann;
    sizeVector_Type M_extBoundary;
};
\end{Code}

\begin{Code}[caption={Typedef per la Classe BC}]
/*! Elemento BC */
typedef BC BC_Type;
/*! Puntatore alla classe BC */												
typedef boost::shared_ptr<BC> BCPtr_Type;
/*! Vettore di puntatori a BC */			
typedef std::vector<BCPtr_Type> BCPtrContainer_Type;	

\end{Code}


\section{Class BCHandler}

in progress

\section{Class DarcyFracture}

in progress

