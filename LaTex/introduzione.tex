\chapter*{Introduzione}

Il presente progetto si inserisce nell'ambito dello studio del moto di un fluido in un mezzo poroso. Lo scopo è quello di studiare e risolvere il problema nel caso di un network di fratture che si intersecano. \\
Il nostro progetto parte da un codice già esistente in C++ in grado di risolvere il problema di Darcy nel caso in cui due fratture abbiano un'intersezione a \textit{X}.
Il nostro obiettivo è quello di estendere il codice ad altri tipi di intersezione, in particolare vogliamo risolvere il problema in presenza di tre fratture che si intersechino in un unico pinto comune formando una \textit{Y}.\\
Per l'implementazione e la risoluzione del problema tramite elementi finiti, abbiamo utilizzato principalmente la libreria \texttt{GetFEM++}, una serie di liberie scritte in C++ che mettono a disposizione dell'utente una serie di funzioni utilizzabili per la soluzione dei problemi FEM. \\
Il modellofisico  su cui si basa la nostra implementazione è un modello ridotto, considera cioè le fratture come delle grandezze 1d, trascurandone lo spessore. Per poter ricavare le condizioni d'interfaccia nel caso della nuova intersezione si è reso necessario ritornare al modello 2d, e quindi considerare l'intersezione come un triangolo e non come un punto. 